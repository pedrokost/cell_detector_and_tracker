\documentclass[12pt,a4paper,openany]{book}

%Uporabljeni paketi
\usepackage[utf8]{inputenc}
\usepackage{cmap}
\usepackage{type1ec}
\usepackage[T1]{fontenc}
\usepackage{fancyhdr}
\usepackage{graphicx,epsfig}
\usepackage[english]{babel}
\usepackage{cite}

\usepackage[pdftex,colorlinks,citecolor=black,filecolor=black,linkcolor=black,urlcolor=black,pagebackref]{hyperref}
\usepackage{tikz}
\bibliographystyle{ieeetr}

%Velikost strani - dvostransko
\oddsidemargin 1.4cm
\evensidemargin 0.35cm
\textwidth 14cm
\topmargin 0.26cm
\headheight 0.6cm
\headsep 1.5cm
\textheight 20cm

%Nastavitev glave in repa strani
\pagestyle{fancy}
\fancyhead{}
\renewcommand{\chaptermark}[1]{\markboth{\textsf{Chapter \thechapter:\ #1}}{}}
\renewcommand{\sectionmark}[1]{\markright{\textsf{\thesection\  #1}}{}}
\fancyhead[RE]{\leftmark}
\fancyhead[LO]{\rightmark}
\fancyhead[LE,RO]{\thepage}
\fancyfoot{}
\renewcommand{\headrulewidth}{0.0pt}
\renewcommand{\footrulewidth}{0.0pt}

\newcommand{\gnuplot}{\textbf{gnuplot}}
\newcommand{\pgfname}{\textsc{pgf}}
\newcommand{\tikzname}{Ti\emph{k}Z}

\input{cc}

%********************************************

\begin{document}

% stran 1 med uvodnimi listi
\thispagestyle{empty} 

\begin{center}
{\large 
IMPERIAL COLLEGE LONDON
DEPARTMENT OF COMPUTING\\
}

\vspace{3cm}
{\LARGE Pedro Kostelec}\\

\vspace{2cm}
\textsc{\textbf{\LARGE 
Cell Tracking methods 
}}

\vspace{2cm}
{ LITERATURE SURVEY }

\vspace{2cm} 
{\Large Supervisor: Ben Glocker}

\vfill
{\Large London, 2014}
\end{center}


%********************************************

% stran 2 med uvodnimi listi
\thispagestyle{empty}

\newpage


%********************************************


\renewcommand\thepage{} 
\tableofcontents 
\renewcommand\thepage{\arabic{page}}

\thispagestyle{empty}

\chapter{Introduction}
This document is report describing the background research that has been completed in preparation for the work on the project entitled ``Automatic Cell Tracking and Categorisation for Microscopic Image Analysis''.

Quantitative analysis of cell populations using time-lapse microscopy is important for understanding cell behaviour. This analysis requires tracking a large number of cells in often low quality image sequences of varying contrast. Often then most effective way to do this is to manually annotate each cell in every frame of the sequence. However, this is slow and tedious, thus limiting the amount of data that we are able to analyse. Several computer vision algorithms have been developed to reduce the amount of manual work required for the analysis of large time-lapse sequences. Ideally, we would like to develop a general algorithm that is able to track any type of cells in image sequences, regardless of the imaging technique used to capture them. However, the current state-of-the-art in computer vision is unable to perform this task accurately and automatically. For this reason, several algorithms have been developed to handle specific cases, relying on heuristics to improve their performance.

The main emphasis of this project is to track leukocytes in low quality image sequences acquired with fluorescent reporter technology and light microscopy. Additionally, the behaviour of these cells with be quantified to allow for further studies of leukocytes.

This is a joint project with Dr. Leo Carlin from the Leukocyte Biology Section at the National Heart and Lung Institute (NHLI) and is supervised by Dr. Ben Glocker.

The remaining of the report is divided into two chapters. In the Chapter \ref{chap:methodoverview} summarize the essential material related to the subject of the project. In the Chapter \ref{chap:workplan} we present an informative work plan that will guide the development of the software. 



\chapter{Cell tracking methods}
\label{chap:methodoverview}

This chapter is an overview of the background research that has been completed in preparation for the work on the project. The chapter is divided into four section. Section \ref{sec:detection} describes methods to perform cell detection on images or sequences of cell images. Section \ref{sec:mitosis} describes the importance of mitosis detection and algorithms that perform it. Section \ref{sec:tracking} presents methods used to track cells in a sequence of images. Finally, in Section \ref{sec:conclusionmethods} we describe which methods seem most promising and discuss reasons why the use of automated cell tracking methods is not as widespread as we would be led to believe, given the wide range of research that has been done on the subject.


\section{Cell detection}
\label{sec:detection}

TODO: overview of types

- watershed
- improved wateshed + merging over fragmented cells
- level sets
- fast level sets
- image restoration, then thresholding
- active contours
- morphological rolling-ball filtering and bayesian classifier
- adaptive thresholding, filtering heuristics
- detection with extremal regions

\subsection{Cell segmentation using the Watershed technique}

A basic cell detection method relies in binarizing an image to separate the background from the cells, followed by a segmentation step to extract the cells. \cite{chen99} approach is as follows:

\begin{enumerate}
  \item Apply a spatial adaptive filter to the image to minimize the effect of noise.
  \item Locate the pixels with minimum intensities in a small sliding window.
  \item For each minimum point we proceed to the progressive flooding of its neighbouring points
  \item Post-processing step to discard false regions.
\end{enumerate}

A more modern, yet similar, approach would perform binarization with Otsu's method \cite{otsu79}, followed by some morphological operations \cite{serra83} to fill holes and eliminate patches that are too small to correspond to cells, and finally the Watershed algorithm \cite{beucher79} to segment the binary image into individual cells. The disadvantage of this method is that the number of pixels belonging to either cells or background should be approximately the same, and that the signal-to-noise ratio is low.

Chen et al \cite{chen06} have used an improved Watershed algorithm \cite{vincent93} to separate cell nuclei after using Otsu's thresholding method to segment nuclei from the background. Additionally they developed a nuclei-fragment merging method based on he shapes and sizes of the nuclei to deal with the problem of over-segmentation caused by the Watershed algorithm.


\subsection{Cell segmentation using level sets}

Another interesting technique to segment cells is a contour evolution method that makes use of the general appearance of cells to segment them using level sets. Mukherjee \cite{mukherjee04} makes the observation that leukocyte shapes and nearly circular cells and that at least a significant part of the border of the cell, the intensity profile is different from the cell cytoplasm and from the background. Using this observation, identification of a leukocyte is formulated as a minimization of an energy function incorporating image gradient and intensity homogeneity within the closed contour encompassing the cell. The benefit of this method is that it can be adjusted to perform well in images with significant clutter and poor contrast by increasing the importance of the homogeneity, or for images with good contrast, where the gradient magnitude term is given more importance. The disadvantage of this method is that cells cannot overlap, which is obtained by adding an additional term to the energy function. The energy function can be minimized with the gradient descent method.

To reduce the solution space for the energy function, only the boundaries of connected components within the image-levels sets. Only the connected components satisfying the size and shape constraints of the cells are extracted. The remaining components are eliminated using area morphology operations. This level-set analysis provides a more efficient solution that is linear in the number of intensity levels in the image in contrast to the much higher complexity of a curve evolution method.

The level set method is contour evolution approach which has good results in segmentation. Tang et al \cite{Tang??} have successfully level-sets combined with local grey thresholding \cite{xu10} for neuron stem cells images which have been obtained by confocal microscopy.

\subsection{Cell detection by model learning}

The previous methods perform efficiently in cells with sufficiently good contrast. In images where the cell borders are unclear, images are of varying intensity, cell density is high, or cells can be of different shapes, these methods would not perform as well. In such cases machine learning methods can perform better by learning a model of a cell based on a large number of annotated examples.

Arteta et al \cite{arteta12}\cite{arteta13},  propose an algorithm that uses a highly-efficient MSER region detector \cite{matas02} to find a broad number of candidate regions that are then scored depending on the similarity to the cell type of interest by a machine learning algorithm . 

The authors organized the extremal regions into trees, so that each tree corresponds to a set of overlapping extremal regions. The non-overlapping regions which achieve high scores can then be selected using dynamic programming of the trees. Two learning strategies are tested: a binary classifier using Support Vector Machines (SVM) and structured learning (structured SVM \cite{joachims09}) which is able to take into account the non-overlap constraint, and achieves better performance.

The feature vector for each regions is composed of several concatenated histograms: a histogram of intensities within the region, two histograms of differences in intensities betwenen the region border and a dilation of it for two different dilation radii, a shape descriptor and the area of the region. The downside of this approach is that extracting the feature vector from the image is slow, especially because it needs to be extracted for each image in the sequence.

The advantage of this approach is the tolerance to changes in image intensities, cell densities and sizes. The major downside is the non-overlap constraint. Fortunately the authors have also developed an algorithm to detect partially overlapping cells \cite{arteta13}. 

The idea is to learn to detect overlapping cells, and the number of cells in the region. The algorithm starts by generating a set of nested regions. Each region is then scored using a set of classifiers that evaluate the similarity of the region to each of the possible classes, where each class corresponds to the number of cells that the region contains. An inference procedure then selects the non-overlapping subset of regions, and assigns each a class label indicating the number of cells that the model believes lie in the region. 

\subsection{Cell detection by image restoration}

Another approach is to use an image restoration technique followed by thresholding \cite{bise11} \cite{huh13}. Bise et al \cite{bise11} apply this technique on phase-contrast microscopy images . The technique utilizes the optophysical principle of image formation by phase-contrast microscope to transform the image into an artefact free image by minimizing a regularized quadratic cost function. After this, a simple image thresholding technique can be used for segmentation.

\section{Mitosis detection}
\label{sec:mitosis}
\section{Cell tracking}
\label{sec:tracking}
\section{Conclusion}
\label{sec:conclusionmethods}

\chapter{Work plan}
\label{chap:workplan}

In order to complete the project in due time, a work plan is outlined to guide the development of the software.

The first phase of the project was background literature research and an overview of the learned methods and techniques is outlined in the previous chapter.

Developing methods to accurately track cells in image sequences is highly data dependent. Before deciding which algorithms we will use in the final project, we will need to review a broad set of sample images in order to analyse their quality. Unfortunately, few have been available to date, and for this reason a broader aspect of methods has been studied than was required. Sample data will also permit further specialization of the study area, which means that more relevant papers will be studied. Dr. Leo Carlin has recently made available a sample image sequence, which has allowed us to estimate the image quality of expected image sequences. He will also send us more data when it is available.

The following is a rough outline of the work plan. This is a preliminary plan, which will be adjusted as work progresses.

\begin{enumerate}
  \item [May] Focus will be on developing cell detection algorithms. For this purpose a basic annotation program may also be written, which should ease annotating image sequences by clicking on cells.
  \item [June] A basic tracking system that will track cells over time. Additionally, a module to compute interesting statistics of cell tracks will be developed.
  \item[July] Improving the cell detection and tracking algorithms. 
  \item [August] Writing a GUI for the software, testing and writing the final report.
\end{enumerate}

The software will be written in MATLAB.


%********************************************

\newpage


\addcontentsline{toc}{chapter}{Bibliography}
\label{page_bibliography}
\bibliography{survey} 



\end{document}




